%%%%%%%%%%%%%%%%%%%%%%%%%%%%%%%%%%%%%%%%%
% Medium Length Professional CV
% LaTeX Template
% Version 2.0 (8/5/13)
%
% This template has been downloaded from:
% http://www.LaTeXTemplates.com
%
% Original author:
% Trey Hunner (http://www.treyhunner.com/)
%
% Important note:
% This template requires the resume.cls file to be in the same directory as the
% .tex file. The resume.cls file provides the resume style used for structuring the
% document.
%
%%%%%%%%%%%%%%%%%%%%%%%%%%%%%%%%%%%%%%%%%

%----------------------------------------------------------------------------------------
%	PACKAGES AND OTHER DOCUMENT CONFIGURATIONS
%----------------------------------------------------------------------------------------


\documentclass{resume} % Use the custom resume.cls style

\usepackage[left=0.75in,top=0.6in,right=0.75in,bottom=0.6in]{geometry} % Document margins

\usepackage[utf8]{inputenc} % Tildes

\usepackage{hyperref} % URLs

\name{Teodoro Freund} % Your name
\address{Washington 2329, Buenos Aires, Argentina} % Your address
\address{tfreund95@gmail.com \\ github.com/teofr}
%\address{123 Pleasant Lane \\ City, State 12345} % Your secondary addess (optional)
%\address{(000)~$\cdot$~111~$\cdot$~1111 \\ john@smith.com} % Your phone number and email

\begin{document}

%----------------------------------------------------------------------------------------
%	EDUCATION SECTION
%----------------------------------------------------------------------------------------

\begin{rSection}{Education}

{\bf Universidad de Buenos Aires} \hfill {\em August 2015 - Present} \\ 
Licenciatura en Ciencias de la Computación \\ (equivalent
to B.Sc. + M.Sc. in Computer Science) \hfill {\em Expected graduation: 2019} \\
Current GPA: 9.71 \hfill{\em On a 10 point scale} 


\begin{rSubsection}{Argentina Training Camp}{July 2016}{}{}
Attended a 2 week intensive camp on programming challenges algorithms and data structures. \footnote{ \url{https://sites.google.com/site/trainingcampargentina2016/english} }
\end{rSubsection}



\begin{rSubsection}{Academic achievements and activities}{}{}{}
\item Competed on the ACM - ICPC South America/South regional contest 2016 and ended 7th (2nd on our university). \footnote{Our team name is El Bueno, el Ralo y el Feo, which means The Good, the Sparse and the Ugly. \\ \url{https://icpc.baylor.edu/regionals/finder/southamerica-south-2016/standings} }

\item Active participation since March 2016 on a reading group about $\lambda$-calculus, rewriting, type theory and much more. \footnote{A translation can be made upon request.  \url{https://sites.google.com/site/labandalambda/} }

\end{rSubsection}

\end{rSection}

%----------------------------------------------------------------------------------------
%	WORK EXPERIENCE SECTION
%----------------------------------------------------------------------------------------

\begin{rSection}{Experience}


\begin{rSubsection}{Universidad de Buenos Aires}{August 2017 - Present}{Teacher Assistant, Haskell course}{}
\item Currently working as a TA on an introductory course on Haskell and functional programming.
\end{rSubsection}

\begin{rSubsection}{Universidad de Buenos Aires}{March 2017 - July 2017}{Teacher Assistant, Algorithms and Data Structures II}{}
\item Worked as a TA on a Data structures and algorithms course in my university.
\item We teach an introduction on algorithms and data structures and a refinement on formal software specification and correctness.
\end{rSubsection}

\end{rSection}




%----------------------------------------------------------------------------------------
%	Projects
%----------------------------------------------------------------------------------------

\begin{rSection}{Projects}

\begin{rSubsection}{Small 32bit OS}{June 2017}{}{}
As an assignment for a subject on computer organization, two other friends and myself made a small OS that could run up to 16 different tasks from 2 different users in protected mode, the user could launch new tasks from the keyboard and the tasks would move through memory eating each other. \footnote{No kidding.}
\end{rSubsection}

\end{rSection}

%----------------------------------------------------------------------------------------

%----------------------------------------------------------------------------------------
%	TECHNICAL STRENGTHS SECTION
%----------------------------------------------------------------------------------------

\begin{rSection}{Technical Strengths}


\begin{tabular}{ @{} >{\bfseries}l @{\hspace{6ex}} l }
Advanced			& C++, C \\
Intermediate		& Haskell, Assembler \\
Basic				& \LaTeX, Git, Linux, Python, OCaml \\
% Computer Languages & C++, C, ASM, Haskell \\
% Tools & Git, \LaTeX
\end{tabular}

\end{rSection}


%----------------------------------------------------------------------------------------
%	EXAMPLE SECTION
%----------------------------------------------------------------------------------------

%\begin{rSection}{Section Name}

%Section content\ldots

%\end{rSection}

%----------------------------------------------------------------------------------------

\end{document}
