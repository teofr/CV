%%%%%%%%%%%%%%%%%%%%%%%%%%%%%%%%%%%%%%%%%
% Medium Length Professional CV
% LaTeX Template
% Version 2.0 (8/5/13)
%
% This template has been downloaded from:
% http://www.LaTeXTemplates.com
%
% Original author:
% Trey Hunner (http://www.treyhunner.com/)
%
% Important note:
% This template requires the resume.cls file to be in the same directory as the
% .tex file. The resume.cls file provides the resume style used for structuring the
% document.
%
%%%%%%%%%%%%%%%%%%%%%%%%%%%%%%%%%%%%%%%%%

%----------------------------------------------------------------------------------------
%	PACKAGES AND OTHER DOCUMENT CONFIGURATIONS
%----------------------------------------------------------------------------------------


\documentclass{resume} % Use the custom resume.cls style

\usepackage[left=0.6in,top=0.6in,right=0.6in,bottom=0.6in]{geometry} % Document margins

\usepackage[utf8]{inputenc} % Tildes

\usepackage{hyperref} % URLs

\name{Teodoro Freund} % Your name
%\address{Washington 2329, Buenos Aires, Argentina} % Your address
\address{tfreund95@gmail.com \\ \url{github.com/teofr}}
%\address{123 Pleasant Lane \\ City, State 12345} % Your secondary addess (optional)
%\address{(000)~$\cdot$~111~$\cdot$~1111 \\ john@smith.com} % Your phone number and email



\def \LANGEN {en}
\def \LANGES {es}

%\def \LANG {es}


\newcommand{\langs}[2]
{\ifx \LANG \LANGEN
	#1
\else
	#2
\fi}


\begin{document}


%----------------------------------------------------------
%	MULTIPLE LANGUAGE MAGIC
%----------------------------------------------------------

\ifx \LANG \undefined
	\def \LANG {en}
\fi


%----------------------------------------------------------------------------------------
%	EDUCATION SECTION
%----------------------------------------------------------------------------------------

\begin{rSection}{\langs{Education}{Educación}}

{\bf Universidad de Buenos Aires} \hfill
{\langs{August 2015 - Present}{Agosto 2015 - Presente}} \\
Licenciatura en Ciencias de la Computación
\langs{\\ (equivalent to B.Sc. + M.Sc. in Computer Science)}{} \hfill
{\em \langs{Expected graduation: December 2020}{Graduación estimada: 2020}} \\
\langs{Current GPA}{Promedio Actual}: 9.71 \langs{\hfill{\em On a 10 point scale}}{}


\begin{rSubsection}
{\langs{Extra university activities}{}}{}{}{}


\item \langs{Active participation since March 2016 on a reading group about $\lambda$-calculus, rewriting, type theory and other subjects.
\footnote{A translation can be made upon request.  \url{https://labandalambda.github.io/} }}
{Pariticipación activa desde Marzo del 2016 en un grupo de lectura sobre cálculo-$\lambda$, reescritura, teoría de tipos y otros temas.
\footnote{\url{https://labandalambda.github.io/} }}

\end{rSubsection}

\begin{rSubsection}
{\langs{Extra courses}{}}{}{}{}

\item Rewriting, lambda calculus and explicit substitutions - \textit{Alejandro Ríos}
\item Randomness and Automata - \textit{Verónica Becher}
\item Topics in Automata Theory - \textit{Olivier Carton}
\item Haskell/GHC's type system and its extensions - \textit{Guillaume Hoffmann}
\item Validating Critical Systems with PVS - \textit{Mariano Moscato}
\item $\lambda$-Calculus and reasonable cost models - \textit{Beniamino Accatoli}
\item Formal Verification with F* and Meta-F* - \textit{Nikhil Swamy, Guido Martinez}



\end{rSubsection}




\end{rSection}

%----------------------------------------------------------------------------------------
%	WORK EXPERIENCE SECTION
%----------------------------------------------------------------------------------------

\begin{rSection}{\langs{Experience}{Experiencia}}

\begin{rSubsection}{Tweag I/O}
{\langs{August 2019 - January 2020}{Agosto 2019 - Enero 2020}}
{\langs{Software Engineer Intern}{Pasante en ingenieria de software}}
{}
\item[] \langs{Currently doing a Software Engineer Internship at Tweag I/O\footnote{\url{https://www.tweag.io/}}. Working on prototyping a new version of a functional programming language, with the main new feature being a gradual typing and contract-based runtime type verification system.  Some of the techniques I'm using to prototype the language can be found in my blog post \url{https://www.tweag.io/posts/2019-11-28-pcf-makam-spec}.}
{Actualmente haciendo una pasatía en ingeniería de Software en Tweag I/O\footnote{\url{https://www.tweag.io/}}. Trabajando en prototipar una nueva versión de un lenguaje de programación funcional, donde la principal novedad es agregarle un sistema de tipos gradual con verificación de tipos en tiempo de ejecución usando contratos. Algunas de las técnicas que utilizo se pueden leer en mi blog post \url{https://www.tweag.io/posts/2019-11-28-pcf-makam-spec}.}
\end{rSubsection}

\begin{rSubsection}{Facebook}
{\langs{January 2019 - March 2019}{Enero 2019 - Marzo 2019}}
{\langs{Software Engineer Intern}{Pasante en ingenieria de software}}
{}
\item[] \langs{Completed a Software Engineer Internship at Facebook during the Winter of 2019. Extended a real time cache invalidator service, that originally worked with only one database, to be able to handle many different ones at the same time. And implemented the invalidator for an internal \texttt{key:value} database as a proof of concept.}
{Completó una pasantía en Facebook como ingeniero de software, durante el verano de 2019. Trabajó extendiendo un servicio de invalidación de cache en tiempo real para soportar más de un tipo de bases de datos al mismo tiempo. Además, dio soporte a una base de datos interna de tipo \texttt{key:value}.}
\end{rSubsection}

\begin{rSubsection}{Google}
{\langs{May 2018 - July 2018}{Mayo 2018 - Julio 2018}}
{\langs{Software Engineer Intern}{Pasante en ingenieria de software}}
{}
\item[] \langs{Completed a Software Engineer Internship at Google during the Summer 2018. Worked on an A/B performance test for a server responsible of authorizing and optimizing reads from a database, the project lowered the noise on these tests from 20\% to 5\%.}
{Completó una pasantía en Google como ingeniero de software, durante el invierno de 2018. Trabajó en un test de performance A/B para un servidor responsable de autorizar y optimizar lecturas a una base de datos, el proyecto disminuyó el ruido en estos tests de un 20\% a un 5\%.}
\end{rSubsection}


\begin{rSubsection}{Universidad de Buenos Aires}
{\langs{July 2017 - December 2017}{Marzo 2017 - Julio 2017}}
{\langs{Teacher Assistant, Algebra Workshop}
{Ayudante de segunda, Algoritmos y Estructuras de Datos II}}{}
\item[] \langs{Worked as a TA on a workshop that students take side by side with the Algebra subject. We teach an introduction to programming using Haskell, and show some concepts related to what they see on the main course (for instance, the relationship between induction and recursion).}{}
\end{rSubsection}

\begin{rSubsection}{Universidad de Buenos Aires}
{\langs{March 2017 - July 2017}{Marzo 2017 - Julio 2017}}
{\langs{Teacher Assistant, Algorithms and Data Structures II}
{Ayudante de segunda, Algoritmos y Estructuras de Datos II}}{}
\item[] \langs{Worked as a TA on a data structures and algorithms course in my university. We teach an introduction on algorithms and data structures and a refinement on formal software specification and correctness.}{Trabajó como ayudante en una materia sobre algoritmos y estructuras de datos. Se enseña una introducción a distintos algoritmos y estructuras de datos y un refinamiento sobre correctitud y especificación formal de software.}
\end{rSubsection}


\end{rSection}


%----------------------------------------------------------------------------------------
%	Honors
%----------------------------------------------------------------------------------------

\begin{rSection}{\langs{Honors and Awards}{}}



\item \langs{Competed on the ACM - ICPC South America regional contest on years 2016, 2017 and 2018, reaching the \textbf{ICPC World Finals} on 2019\footnote{Our team name is El Bueno, el Ralo y el Feo, which translates to The Good, the Sparse and the Ugly. \\
\url{https://icpc.baylor.edu/ICPCID/MPJ2YWYRR35P}}}
{Compitió en las regionals de Sudamérica de la ACM - ICPC en los años 2016, 2017 and 2018, alcanzando y compitiendo en la 43 ICPC World Finals en 2019\footnote{El nombre de nuestro equipo es El Bueno, el Ralo y el Feo. \\
\url{https://icpc.baylor.edu/ICPCID/MPJ2YWYRR35P}}}


\end{rSection}


%----------------------------------------------------------------------------------------
%	Projects
%----------------------------------------------------------------------------------------

%\begin{rSection}{\langs{Projects}{Proyectos}}

%\begin{rSubsection}{\langs{Small 32bit OS}{Pequeño SO de 32bits}}
%{\langs{June 2017}{Junio 2017}}{}{}

%\langs{As an assignment for a subject on computer organization, two other friends and myself made a small OS that could run concurrently up to 16 different tasks from 2 different users in protected mode.}
%{Cómo un trabajo para una materia sobre organización del computador, con dos compañeros hicimos un pequeño SO capaz de ejecutar hasta 16 tareas concurrentemente de 2 usuarios distintos en modo protegido.}
%\end{rSubsection}

%\end{rSection}

%----------------------------------------------------------------------------------------

%----------------------------------------------------------------------------------------
%	TECHNICAL STRENGTHS SECTION
%----------------------------------------------------------------------------------------


\begin{rSection}{\langs{Technical Strengths}{Habilidades técnicas}}


\begin{tabular}{ @{} >{\bfseries}l @{\hspace{6ex}} l }
\langs{Advanced}{Avanzado}			& C++, C \\
\langs{Intermediate}{Intermedio}	& Haskell, Assembler, Python, Rust, Git \\
\langs{Basic}{Básico}				& \LaTeX, Linux, Coq, PVS, F*, Prolog

\end{tabular}

\end{rSection}

%----------------------------------------------------------------------------------------
%	LANGUAGES SECTION
%----------------------------------------------------------------------------------------


\begin{rSection}{\langs{Languages}{Idiomas}}

\begin{center}
\begin{tabular}{ @{} >{\bfseries}l @{\hspace{6ex}} r || @{\hspace{1ex}} >{\bfseries}l @{\hspace{6ex}} r }
\langs{Written English}{Inglés escrito}			& \langs{Proficient}{Fluido} &
\langs{Spoken English}{Inglés hablado}			& \langs{Proficient}{Fluido} \\
\langs{Spanish}{Español}		& \langs{Native}{Nativo} & 
\langs{Italian}{Italiano}				& \langs{Intermediate}{Intermedio} \\
\langs{French}{Francés} & \langs{Basic}{Básico} &
\langs{German}{Alemán} & \langs{Basic}{Básico}

\end{tabular}
\end{center}

\end{rSection}


%----------------------------------------------------------------------------------------
%	EXAMPLE SECTION
%----------------------------------------------------------------------------------------

%\begin{rSection}{Section Name}

%Section content\ldots

%\end{rSection}

%----------------------------------------------------------------------------------------

\end{document}
