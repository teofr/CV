%%%%%%%%%%%%%%%%%%%%%%%%%%%%%%%%%%%%%%%%%
% Medium Length Professional CV
% LaTeX Template
% Version 2.0 (8/5/13)
%
% This template has been downloaded from:
% http://www.LaTeXTemplates.com
%
% Original author:
% Trey Hunner (http://www.treyhunner.com/)
%
% Important note:
% This template requires the resume.cls file to be in the same directory as the
% .tex file. The resume.cls file provides the resume style used for structuring the
% document.
%
%%%%%%%%%%%%%%%%%%%%%%%%%%%%%%%%%%%%%%%%%

%----------------------------------------------------------------------------------------
%	PACKAGES AND OTHER DOCUMENT CONFIGURATIONS
%----------------------------------------------------------------------------------------


\documentclass{resume} % Use the custom resume.cls style

\usepackage[left=0.75in,top=0.6in,right=0.75in,bottom=0.6in]{geometry} % Document margins

\usepackage[utf8]{inputenc} % Tildes

\usepackage{hyperref} % URLs

\name{Teodoro Freund} % Your name
\address{Washington 2329, Buenos Aires, Argentina} % Your address
\address{tfreund95@gmail.com \\ github.com/teofr}
%\address{123 Pleasant Lane \\ City, State 12345} % Your secondary addess (optional)
%\address{(000)~$\cdot$~111~$\cdot$~1111 \\ john@smith.com} % Your phone number and email



\def \LANGEN {en}
\def \LANGES {es}

%\def \LANG {es}


\newcommand{\langs}[2]
{\ifx \LANG \LANGEN
	#1
\else
	#2
\fi}


\begin{document}


%----------------------------------------------------------
%	MULTIPLE LANGUAGE MAGIC
%----------------------------------------------------------

\ifx \LANG \undefined
	\def \LANG {en}
\fi


%----------------------------------------------------------------------------------------
%	EDUCATION SECTION
%----------------------------------------------------------------------------------------

\begin{rSection}{\langs{Education}{Educación}}

{\bf Universidad de Buenos Aires} \hfill
{\langs{August 2015 - Present}{Agosto 2015 - Presente}} \\
Licenciatura en Ciencias de la Computación
\langs{\\ (equivalent to B.Sc. + M.Sc. in Computer Science)}{} \hfill
{\em \langs{Expected graduation: 2020}{Graduación estimada: 2020}} \\
\langs{Current GPA}{Promedio Actual}: 9.6 \langs{\hfill{\em On a 10 point scale}}{}


%\begin{rSubsection}{Argentina Training Camp}
%{\langs{July 2016}{Julio 2016}}{}{}
%
%\langs{Attended a 2 week intensive camp on programming challenges algorithms and data structures.}
%{Asistió un curso intensivo de 2 semanas sobre algoritmos y estructuras de datos en programación competitiva.}
%\footnote{ \langs{\url{https://sites.google.com/site/trainingcampargentina2016/english}}{\url{https://sites.google.com/site/trainingcampargentina2016}} }
%\end{rSubsection}



\begin{rSubsection}
{\langs{Academic achievements and activities}{Actividades y logros académicos}}{}{}{}

\item \langs{Active participation since March 2016 on a reading group about $\lambda$-calculus, rewriting, type theory and other subjects.
\footnote{A translation can be made upon request.  \url{https://sites.google.com/site/labandalambda/} }}
{Pariticipación activa desde Marzo del 2016 en un grupo de lectura sobre cálculo-$\lambda$, reescritura, teoría de tipos y otros temas.
\footnote{\url{https://sites.google.com/site/labandalambda/} }}

\item \langs{Competed on the ACM - ICPC 2017 South America/South regional contest and ended 15$^{th}$ (3$^{rd}$ on our university).
\footnote{Our team name is El Bueno, el Ralo y el Feo, which translates to The Good, the Sparse and the Ugly. \\
\url{https://icpc.baylor.edu/regionals/finder/southamerica-south-2017/standings} }}
{Compitió en la regional de Sudamérica/Sur de la competencia ACM - ICPC 2017 y finalizó en la $15^{a}$ posición ($3^{o}$ dentro de la universidad).
\footnote{Nuestro equipo se llamó {\em El Bueno, el Ralo y el Feo} \\
\url{https://icpc.baylor.edu/regionals/finder/southamerica-south-2017/standings} }}

\item \langs{Competed on the ACM - ICPC 2016 South America/South regional contest and ended 7$^{th}$ (2$^{nd}$ on our university).
\footnote{\url{https://icpc.baylor.edu/regionals/finder/southamerica-south-2016/standings} }}
{Compitió en la regional de Sudamérica/Sur de la competencia ACM - ICPC 2016 y finalizó en la $7^{ma}$ posición ($2^{do}$ dentro de la universidad).
\footnote{\url{https://icpc.baylor.edu/regionals/finder/southamerica-south-2016/standings} }}

\item \langs{Attended a 2 week intensive camp on programming challenges algorithms and data structures during July 2016.}
{Asistió un curso intensivo de 2 semanas sobre algoritmos y estructuras de datos en programación competitiva durante Julio del 2016.}
\footnote{ \langs{\url{https://sites.google.com/site/trainingcampargentina2016/english}}{\url{https://sites.google.com/site/trainingcampargentina2016}} }

\end{rSubsection}

\end{rSection}

%----------------------------------------------------------------------------------------
%	WORK EXPERIENCE SECTION
%----------------------------------------------------------------------------------------

\begin{rSection}{\langs{Experience}{Experiencia}}

\begin{rSubsection}{Google}
{\langs{May 2018 - July 2018}{Mayo 2018 - Julio 2018}}
{\langs{Software Engineer Intern}{Pasante en ingenieria de software}}
{}
\item[] \langs{Currently doing a Software Engineering Internship at Google for the Summer 2018.}
{Actualmente haciendo una pasantía en Google como ingeniero de software, durante el invierno de 2018.}
\end{rSubsection}


\begin{rSubsection}{Universidad de Buenos Aires}
{\langs{August 2017 - December 2017}{Agosto 2017 - Diciembre 2017}}
{\langs{Teacher Assistant, Haskell course}
{Ayudante de segunda, introducción a Haskell}}
{}
\item[] \langs
{Worked as a TA on an introductory course on Haskell and functional programming.}
{Trabajó como ayudante en un curso introductorio sobre Haskell y programación funcional.}
\end{rSubsection}

\begin{rSubsection}{Universidad de Buenos Aires}
{\langs{March 2017 - July 2017}{Marzo 2017 - Julio 2017}}
{\langs{Teacher Assistant, Algorithms and Data Structures II}
{Ayudante de segunda, Algoritmos y Estructuras de Datos II}}{}
\item[] \langs{Worked as a TA on a data structures and algorithms course in my university. We teach an introduction on algorithms and data structures and a refinement on formal software specification and correctness.}{Trabajó como ayudante en una materia sobre algoritmos y estructuras de datos. Se enseña una introducción a distintos algoritmos y estructuras de datos y un refinamiento sobre correctitud y especificación formal de software.}
\end{rSubsection}

\end{rSection}




%----------------------------------------------------------------------------------------
%	Projects
%----------------------------------------------------------------------------------------

\begin{rSection}{\langs{Projects}{Proyectos}}

\begin{rSubsection}{\langs{Small 32bit OS}{Pequeño SO de 32bits}}
{\langs{June 2017}{Junio 2017}}{}{}

\langs{As an assignment for a subject on computer organization, two other friends and myself made a small OS that could run concurrently up to 16 different tasks from 2 different users in protected mode.}
{Cómo un trabajo para una materia sobre organización del computador, con dos compañeros hicimos un pequeño SO capaz de ejecutar hasta 16 tareas concurrentemente de 2 usuarios distintos en modo protegido.}
\end{rSubsection}

\end{rSection}

%----------------------------------------------------------------------------------------

%----------------------------------------------------------------------------------------
%	TECHNICAL STRENGTHS SECTION
%----------------------------------------------------------------------------------------


\begin{rSection}{\langs{Technical Strengths}{Habilidades técnicas}}


\begin{tabular}{ @{} >{\bfseries}l @{\hspace{6ex}} l }
\langs{Advanced}{Avanzado}			& C++, C \\
\langs{Intermediate}{Intermedio}		& Haskell, Assembler, Python \\
\langs{Basic}{Básico}				& \LaTeX, Git, Linux, Coq, PVS 

\end{tabular}

\end{rSection}

%----------------------------------------------------------------------------------------
%	LANGUAGES SECTION
%----------------------------------------------------------------------------------------


\begin{rSection}{\langs{Languages}{Idiomas}}

\begin{center}
\begin{tabular}{ @{} >{\bfseries}l @{\hspace{6ex}} r || @{\hspace{1ex}} >{\bfseries}l @{\hspace{6ex}} r }
\langs{Written English}{Inglés escrito}			& \langs{Proficient}{Fluido} &
\langs{Spoken English}{Inglés hablado}			& \langs{Proficient}{Fluido} \\
\langs{Spanish}{Español}		& \langs{Native}{Nativo} & 
\langs{Italian}{Italiano}				& \langs{Intermediate}{Intermedio} \\
\langs{French}{Francés} & \langs{Basic}{Básico} &
\langs{German}{Alemán} & \langs{Basic}{Básico}

\end{tabular}
\end{center}

\end{rSection}


%----------------------------------------------------------------------------------------
%	EXAMPLE SECTION
%----------------------------------------------------------------------------------------

%\begin{rSection}{Section Name}

%Section content\ldots

%\end{rSection}

%----------------------------------------------------------------------------------------

\end{document}
