%%%%%%%%%%%%%%%%%%%%%%%%%%%%%%%%%%%%%%%%%
% Medium Length Professional CV
% LaTeX Template
% Version 2.0 (8/5/13)
%
% This template has been downloaded from:
% http://www.LaTeXTemplates.com
%
% Original author:
% Trey Hunner (http://www.treyhunner.com/)
%
% Important note:
% This template requires the resume.cls file to be in the same directory as the
% .tex file. The resume.cls file provides the resume style used for structuring the
% document.
%
%%%%%%%%%%%%%%%%%%%%%%%%%%%%%%%%%%%%%%%%%

%----------------------------------------------------------------------------------------
%	PACKAGES AND OTHER DOCUMENT CONFIGURATIONS
%----------------------------------------------------------------------------------------


\documentclass{resume} % Use the custom resume.cls style

\usepackage[left=0.6in,top=0.6in,right=0.6in,bottom=0.6in]{geometry} % Document margins

\usepackage[utf8]{inputenc} % Tildes

\usepackage{hyperref} % URLs

\name{Teodoro Freund} % Your name
%\address{Washington 2329, Buenos Aires, Argentina} % Your address
\address{\texttt{tfreund95@gmail.com} \\ \href{https://teofr.github.io/}{\texttt{teofr.github.io}}}
%\address{123 Pleasant Lane \\ City, State 12345} % Your secondary addess (optional)
%\address{(000)~$\cdot$~111~$\cdot$~1111 \\ john@smith.com} % Your phone number and email



\def \LANGEN {en}
\def \LANGES {es}

% \def \LANG {es}


\newcommand{\langs}[2]
{\ifx \LANG \LANGEN
	#1
\else
	#2
\fi}


\begin{document}


%----------------------------------------------------------
%	MULTIPLE LANGUAGE MAGIC
%----------------------------------------------------------

\ifx \LANG \undefined
	\def \LANG {en}
\fi


%----------------------------------------------------------------------------------------
%	EDUCATION SECTION
%----------------------------------------------------------------------------------------

\begin{rSection}{\langs{Education}{Educación}}

{\bf Universidad de Buenos Aires} \hfill
{\langs{August 2015 - Present}{Agosto 2015 - Presente}} \\
Licenciatura en Ciencias de la Computación
\langs{\\ (equivalent to B.Sc. + M.Sc. in Computer Science)}{} \hfill
{\em \langs{Expected graduation: April 2021}{Graduación estimada: Abril 2021}} \\
\langs{Current GPA}{Promedio Actual}: 9.73 \langs{\hfill{\em On a 10 point scale}}{}


%\begin{rSubsection}{Argentina Training Camp}
%{\langs{July 2016}{Julio 2016}}{}{}
%
%\langs{Attended a 2 week intensive camp on programming challenges algorithms and data structures.}
%{Asistió un curso intensivo de 2 semanas sobre algoritmos y estructuras de datos en programación competitiva.}
%\footnote{ \langs{\url{https://sites.google.com/site/trainingcampargentina2016/english}}{\url{https://sites.google.com/site/trainingcampargentina2016}} }
%\end{rSubsection}



\begin{rSubsection}
{\langs{Academic achievements and activities}{Actividades y logros académicos}}{}{}{}

\item \langs{Active participation since March 2016 on a reading group about functional programming, $\lambda$-calculus, rewriting, type theory and related subjects. 
% I'm in charge of the group since March 2020.
\footnote{\href{https://labandalambda.github.io/}{\texttt{labandalambda.github.io}} }}
{Participo activamente desde Marzo del 2016 en un grupo de lectura sobre programción funcional, cálculo-$\lambda$, reescritura, teoría de tipos y otros temas relacionados. Estoy a cargo del grupo desde Marzo del 2020.
\footnote{\href{https://labandalambda.github.io/}{\texttt{labandalambda.github.io}} }}

%\item \langs{Attended a week long course on Haskell/GHC type system and it's extensions during February 2018.}{Asistió un curso intensivo sobre el sistema de tipos de Haskell/GHC y sus extensiones durante el verano de 2018.}
%\footnote{\langs{Spanish version only.}{} \url{https://cs.famaf.unc.edu.ar/~hoffmann/rio18/}}

\item \langs{I competed on the ACM - ICPC South America regional contest on years 2016, 2017 and 2018, reaching the \textbf{ICPC World Finals} on 2019\footnote{Our team name is El Bueno, el Ralo y el Feo, which translates to The Good, the Sparse and the Ugly. \\
\href{https://icpc.baylor.edu/ICPCID/MPJ2YWYRR35P}{\texttt{icpc.baylor.edu/ICPCID/MPJ2YWYRR35P}}}. I prepared problems for the 2019 and 2020 Argentinian competition.}
{Competí en las regionals de Sudamérica de la ACM - ICPC en los años 2016, 2017 y 2018, alcanzando y compitiendo en la \textbf{ICPC World Finals} en 2019\footnote{El nombre de nuestro equipo es El Bueno, el Ralo y el Feo. \\
\href{https://icpc.baylor.edu/ICPCID/MPJ2YWYRR35P}{\texttt{icpc.baylor.edu/ICPCID/MPJ2YWYRR35P}}}}

%\item \langs{Attended a 2 week intensive camp on programming challenges algorithms and data structures during July 2016.}
%{Asistió un curso intensivo de 2 semanas sobre algoritmos y estructuras de datos en programación competitiva durante Julio del 2016.}
%\footnote{ \langs{\url{https://sites.google.com/site/trainingcampargentina2016/english}}{\url{https://sites.google.com/site/trainingcampargentina2016}} }

\end{rSubsection}


\begin{rSubsection}
	{\langs{Extracurricular courses}{}}{}{}{}
	
	\item Rewriting, lambda calculus and explicit substitutions - \textit{Alejandro Ríos}
	\item Randomness and Automata - \textit{Verónica Becher}
	\item Topics in Automata Theory - \textit{Olivier Carton}
	\item Haskell/GHC's type system and its extensions - \textit{Guillaume Hoffmann}
	\item Validating Critical Systems with PVS - \textit{Mariano Moscato}
	\item $\lambda$-Calculus and reasonable cost models - \textit{Beniamino Accatoli}
	\item Formal Verification with F* and Meta-F* - \textit{Nikhil Swamy, Guido Martinez}
	\item Behavioural Types and Contracts - \textit{Hernán Melgratti}
\end{rSubsection}

\end{rSection}

%----------------------------------------------------------------------------------------
%	WORK EXPERIENCE SECTION
%----------------------------------------------------------------------------------------

\begin{rSection}{\langs{Experience}{Experiencia}}

\begin{rSubsection}{Invgate\footnote{\href{https://www.invgate.com/}{\texttt{invgate.com}}}}
{\langs{July 2020 - Present}{Julio 2020 - Presente}}
{\langs{Backend Software Engineer}{Backend Software Engineer}}
{}
\item[] \langs{We develop a product that provides automatic inventory of IT assets. The main challenges involve handling big amounts of data, sweeping and recoverying valuable information from a company's network, providing reliable and easily deployable software to install on all of a company's computers, etc.}{}	
\end{rSubsection}

\begin{rSubsection}{Tweag I/O\footnote{\href{https://www.tweag.io/}{\texttt{tweag.io}}}}
{\langs{August 2019 - January 2020}{Agosto 2019 - Enero 2020}}
{\langs{Software Engineer Intern}{Pasante en ingeniería de software}}
{}
\item[] \langs{I prototyped a new configuration functional programming language, with a novel gradual type system. Some of the techniques I used for rapid experimentation can be found on my blog post\footnote{
\href{https://www.tweag.io/blog/2019-11-28-PCF-makam-spec/}{\texttt{tweag.io/blog/2019-11-28-PCF-makam-spec}}}. And the language can be found on the open repository\footnote{\href{https://github.com/tweag/nickel}{\texttt{github.com/tweag/nickel}}}}
{Prototipé un lenguaje de configuración funcional, con un sistema novedoso de tipos graduales. Algunas de las técnicas que utilicé para experimentar rapidamente se pueden leer en mi blog post\footnote{
\href{https://www.tweag.io/blog/2019-11-28-PCF-makam-spec/}{\texttt{tweag.io/blog/2019-11-28-PCF-makam-spec}}}.}
\end{rSubsection}

\begin{rSubsection}{Facebook}
{\langs{January 2019 - March 2019}{Enero 2019 - Marzo 2019}}
{\langs{Software Engineer Intern}{Pasante en ingeniería de software}}
{}
\item[] \langs{I extended a real time cache invalidator service, that originally worked with only one database, to be able to handle many different ones at the same time. And I implemented the invalidator for an internal \texttt{key:value} database as a proof of concept.}
{Extendí un servicio de invalidación de cache en tiempo real para soportar más de un tipo de bases de datos al mismo tiempo. Además, di soporte a una base de datos interna de tipo \texttt{key:value} como prueba de concepto.}
\end{rSubsection}

\begin{rSubsection}{Google}
{\langs{May 2018 - July 2018}{Mayo 2018 - Julio 2018}}
{\langs{Software Engineer Intern}{Pasante en ingeniería de software}}
{}
\item[] \langs{I built an A/B performance test for a server responsible of authorizing and optimizing queries from a database, the project lowered the noise on these tests from 20\% to 5\%.}
{Hice un test de performance A/B para un servidor responsable de autorizar y optimizar lecturas a una base de datos, el proyecto disminuyó el ruido en estos tests de un 20\% a un 5\%.}
\end{rSubsection}


% \begin{rSubsection}{Universidad de Buenos Aires}
% {\langs{March 2017 - December 2017}{Marzo 2017 - Diciembre 2017}}
% {\langs{Teacher Assistant}
% {Ayudante de segunda}}
% {}
% \item[] \langs
% {I worked as a TA on two courses, an introductory course on Haskell and functional programming. And an algorithms and data structures course, where the main focus was on teaching some classic algorithms and data structures, as well as a basis on formal software specification.}
% {Trabajé como ayudante en dos cursos, un curso introductorio sobre Haskell y programación funcional. Y otro de algoritmos y estructuras de datos, donde el foco principal era estudiar algoritmos y estructuras clásicos, y enseñar las bases de un sistema de especificación formal de software.}
% \end{rSubsection}

\begin{rSubsection}{Universidad de Buenos Aires}
{\langs{July 2017 - December 2017}{Marzo 2017 - Julio 2017}}
{\langs{Teaching Assistant, Algebra Workshop}
{Ayudante de segunda, Taller de Álgebra}}{}
\item[] \langs{Worked as a TA on a workshop that students take side by side with the Algebra subject. We teach an introduction to programming using Haskell, and show some concepts related to what they see on the main course (for instance, the relationship between induction and recursion).}{}
\end{rSubsection}

\begin{rSubsection}{Universidad de Buenos Aires}
{\langs{March 2017 - July 2017}{Marzo 2017 - Julio 2017}}
{\langs{Teaching Assistant, Algorithms and Data Structures II}
{Ayudante de segunda, Algoritmos y Estructuras de Datos II}}{}
\item[] \langs{Worked as a TA on a data structures and algorithms course in my university. We introduce students on some classic algorithms and data structures, with a focus on formal software specification and correctness.}{Trabajó como ayudante en una materia sobre algoritmos y estructuras de datos. Se enseña una introducción a distintos algoritmos y estructuras de datos y un refinamiento sobre correctitud y especificación formal de software.}
\end{rSubsection}

%\begin{rSubsection}{Universidad de Buenos Aires}
%{\langs{March 2017 - July 2017}{Marzo 2017 - Julio 2017}}
%{\langs{Teacher Assistant, Algorithms and Data Structures II}
%{Ayudante de segunda, Algoritmos y Estructuras de Datos II}}{}
%\item[] \langs{Worked as a TA on a data structures and algorithms course in my university. We teach an introduction on algorithms and data structures and a refinement on formal software specification and correctness.}{Trabajó como ayudante en una materia sobre algoritmos y estructuras de datos. Se enseña una introducción a distintos algoritmos y estructuras de datos y un refinamiento sobre correctitud y especificación formal de software.}
%\end{rSubsection}

\end{rSection}




%----------------------------------------------------------------------------------------
%	Projects
%----------------------------------------------------------------------------------------

% \begin{rSection}{\langs{Projects}{Proyectos}}

% \begin{rSubsection}{\langs{Small 32bit OS}{Pequeño SO de 32bits}}
% {\langs{June 2017}{Junio 2017}}{}{}

% \langs{As an assignment for a subject on computer organization, two other friends and myself made a small OS that could run concurrently up to 16 different tasks from 2 different users in protected mode.}
% {Cómo un trabajo para una materia sobre organización del computador, con dos compañeros hicimos un pequeño SO capaz de ejecutar hasta 16 tareas concurrentemente de 2 usuarios distintos en modo protegido.}
% \end{rSubsection}

% \end{rSection}

%----------------------------------------------------------------------------------------

%----------------------------------------------------------------------------------------
%	TECHNICAL STRENGTHS SECTION
%----------------------------------------------------------------------------------------


\begin{rSection}{\langs{Technical Strengths}{Habilidades técnicas}}


\begin{tabular}{ @{} >{\bfseries}l @{\hspace{6ex}} l }
\langs{Advanced}{Avanzado}			& C++, C, Python \\
\langs{Intermediate}{Intermedio}	& Haskell, Assembler, Rust, Git, \LaTeX \\
\langs{Basic}{Básico}				& Linux, Coq, PVS, F*, Prolog, Bash, Racket, Makam, Agda

\end{tabular}

\end{rSection}

%----------------------------------------------------------------------------------------
%	LANGUAGES SECTION
%----------------------------------------------------------------------------------------


\begin{rSection}{\langs{Languages}{Idiomas}}

\begin{center}
\begin{tabular}{ @{} >{\bfseries}l @{\hspace{6ex}} r || @{\hspace{1ex}} >{\bfseries}l @{\hspace{6ex}} r }
\langs{Written English}{Inglés escrito}			& \langs{Proficient}{Fluido} &
\langs{Spoken English}{Inglés hablado}			& \langs{Proficient}{Fluido} \\
\langs{Spanish}{Español}		& \langs{Native}{Nativo} & 
\langs{Italian}{Italiano}				& \langs{Intermediate}{Intermedio} \\
\langs{French}{Francés} & \langs{Basic}{Básico} &
\langs{German}{Alemán} & \langs{Basic}{Básico} \\

\end{tabular}
\end{center}

\end{rSection}

\vspace{3cm}


%----------------------------------------------------------------------------------------
%	EXAMPLE SECTION
%----------------------------------------------------------------------------------------

%\begin{rSection}{Section Name}

%Section content\ldots

%\end{rSection}

%----------------------------------------------------------------------------------------

\end{document}
